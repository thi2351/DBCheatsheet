\cheatbox{Ch2.ERD}{\textStrech}{
\subsubsection*{1. Overview of DB design process}
	\includegraphics[width=0.7\linewidth]{figures/2-1}
\subsubsection*{2. ERD}

\begin{tabular}{|m{1.5cm}|m{1.5cm}|m{3.5cm}|}
	\hline
	\textbf{Illustration} & \textbf{Attribute Type} & \textbf{Description} \\
	\hline
	\includegraphics[width=0.8\linewidth]{figures/2-2}
	& Simple Attribute &
	Has a single, atomic value that cannot be further divided. \\
	\hline
	\begin{center}
	\includegraphics[width=1\linewidth]{figures/2-3}
	\end{center}
	& Composite Attribute &
	Consists of multiple components that can be meaningfully separated. \\
	\hline
	\includegraphics[width=1\linewidth]{figures/2-4}
	& Multi-valued Attribute &
	May have multiple values for a single entity. May have constrain the number of values allowed for each individual identity. \\
	\hline
	\includegraphics[width=1\linewidth]{figures/2-5}
	& Derived Attribute &
	Its value is derived from other related attributes rather than stored directly. \\
	\hline
	\includegraphics[width=1\linewidth]{figures/2-6}
	& Complex Attribute &
	A combination of composite and multi-valued attributes. \\
	\hline
	\includegraphics[width=1\linewidth]{figures/2-7}
	& Key Attribute &
	An attribute (or set of attributes) whose values uniquely identify each entity in an entity set. \\
	\hline
\end{tabular}
\textbf{Degree} of a relationship type: Number of participating entity types. (recursive is unary)


\textbf{Cardinality ratios:} the maximum number of instances an entity can participate in (1:1, 1:n, n:1, n:m).


\textbf{Participation:} the minimum number of relationship instances that each entity can participate in. (Total participation: every entity participates in, Partial: some (not every) entities).

\textbf{Attribute of a relationship type:} Relationship types can also have attributes.
}

\cheatbox{Ch2. ERD}{\textStrech}{
	\subsubsection*{Weak Entities Type}
	\begin{itemize}
		\item A \textbf{weak entity type} does \textbf{not have its own key attribute}.
		\item Weak entities are \textbf{identified} by an \textbf{owner (identifying) entity type}, and an \textbf{identifying relationship}, together with a \textbf{partial key}.
		\item A weak entity type has \textbf{total participation}
		(existence dependency) in its identifying relationship.
		\item A \textbf{partial key} uniquely identifies weak entities
		related to the same owner entity.
	\end{itemize}
	\subsubsection*{3. EERD}
	\begin{itemize}
		\item \textbf{Specialization}:
		process of defining subclasses of an entity type with
		\begin{itemize}
			\item specific attributes,
			\item specific relationship types.
		\end{itemize}
		
		\item \textbf{Generalization}:
		reverse abstraction process that
		\begin{itemize}
			\item identifies common features of multiple entity types,
			\item combines them into a single superclass.
		\end{itemize}

	\end{itemize}
	\centering
	\includegraphics[width=0.48\linewidth]{figures/2-9}
	\hfill
	\includegraphics[width=0.48\linewidth]{figures/2-8}
	
	d: disjoint, o: overlapping

\subsubsection*{Fan trap}

\centering
\includegraphics[width=0.48\linewidth]{figures/2-10}
\hfill
\includegraphics[width=0.48\linewidth]{figures/2-11}


\subsubsection*{Chamsp trap}
\includegraphics[width=0.48\linewidth]{figures/2-12}
\hfill
\includegraphics[width=0.48\linewidth]{figures/2-13}


}