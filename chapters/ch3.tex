\cheatbox{Ch3. Relational Data Model}{\textStrech}{
	\subsubsection*{1. Data Model}
	
	A \textbf{data model} consists of three components:
	\textbf{data structures}, \textbf{operations}, and \textbf{integrity constraints}.
	
	\paragraph*{Data Structures}
	In the relational data model, data is represented using \textbf{relations}.
	
	An \textbf{$n$-ary relation} $R$ defined on domains
	$S_1, S_2, \dots, S_n$ is a subset of their Cartesian product:
	\[
	R \subseteq S_1 \times S_2 \times \cdots \times S_n
	\]
	
	Each row of $R$ is an \textbf{$n$-tuple}:
	\[
	t = \langle v_1, v_2, \dots, v_n \rangle,
	\quad v_i \in S_i
	\]
	
	\textbf{Inherent (implicit) properties}:
	\begin{itemize}
		\item Rows represent $n$-tuples; ordering of rows is immaterial.
		\item All rows are distinct.
		\item Column ordering is significant and corresponds to
		$S_1, S_2, \dots, S_n$.
		\item Each column is labeled by an \textbf{attribute} indicating its domain.
	\end{itemize}
	
	\textbf{Degree and Cardinality}:
	\begin{itemize}
		\item \textbf{Degree}: number of attributes ($n$).
		\item \textbf{Cardinality}: number of tuples ($|R|$).
	\end{itemize}
	
	\paragraph*{Operations}
	Operations specify how data can be retrieved and manipulated.
	Core relational algebra operators:
	\begin{itemize}
		\item \textbf{Selection} ($\sigma$),
		\textbf{Projection} ($\pi$),
		\textbf{Union} ($\cup$),
		\textbf{Set Difference} ($-$),
		\textbf{Cartesian Product} ($\times$),
		\textbf{Join} ($\bowtie$).
	\end{itemize}
	Relational operators satisfy the \textbf{closure property}.
	
	\paragraph*{Integrity Constraints}
	Integrity constraints define valid database states:
\begin{itemize}
	\item \textbf{Domain constraint}: attribute values must belong to the domain  
	(e.g., Age $\ge 0$).
	
	\item \textbf{Key constraint}: a key is a \textbf{minimal superkey};
	key values must be unique  
	(e.g., StudentID identifies STUDENT).
	
	\item \textbf{Constraints on nulls}: specify whether NULL values are permitted
	for attributes.
	
	\item \textbf{Entity integrity constraint}: primary key attributes
	cannot have NULL values.
	
	\item \textbf{Referential integrity constraint}: foreign key values must reference
	a primary key value or be NULL  
	(e.g., STUDENT.DeptID $\rightarrow$ DEPARTMENT.DeptID).
\end{itemize}

}

\cheatbox{Ch3. Relational Data Model}{\textStrech}{
	\subsubsection*{2. Data model mapping}
	\begin{enumerate}
		\item \textbf{Map Regular (Strong) Entity Types}:  
		Create a relation for each strong entity type; include all simple attributes.
		Primary key = entity key.
		
		\item \textbf{Map Weak Entity Types}:  
		Create a relation for the weak entity including its simple attributes.
		Primary key = (partial key + primary key of owner).
		Owner key is a foreign key.
		
		\item \textbf{Map Binary 1:1 Relationships}:  
		Choose one relation (prefer total participation) and include the primary key
		of the other as a foreign key; add relationship attributes.
		
		\item \textbf{Map Binary 1:N Relationships}:  
		Include the primary key of the 1-side as a foreign key in the N-side relation;
		add relationship attributes to the N-side.
		
		\item \textbf{Map Binary M:N Relationships}:  
		Create a new relation whose primary key is the combination of the primary keys
		of participating entities; include relationship attributes.
		
		\item \textbf{Map Multivalued Attributes}:  
		Create a new relation with attributes (attribute value + owner primary key).
		Primary key = both attributes.
		
		\item \textbf{Map n-ary Relationships ($n > 2$)}:  
		Create a new relation including primary keys of all participating entities
		as foreign keys; primary key is their combination.
		
		\item \textbf{Map Specialization / Generalization}:  
		Use one of the following:
		\begin{itemize}
			\item Superclass + subclass relations
			\item Single relation with type attribute
			\item Multiple relations for subclasses only
		\end{itemize}
		
		\item \textbf{Map Categories (Union Types)}:  
		Create a relation with a surrogate key and foreign keys referencing
		each superclass; enforce membership constraint.
	\end{enumerate}
}

\cheatbox{Ch3. Relational Data Model}{\textStrech}{
	\subsubsection*{3. Relational Algebra}
	\renewcommand{\arraystretch}{1.3}
\begin{tabular}{|p{1.75cm}|p{4.0cm}|p{1.25cm}|}
	\hline
	\textbf{Operation} & \textbf{Purpose} & \textbf{Notation} \\
	\hline
	SELECT &
	Selects all tuples that satisfy the selection condition from a relation $R$. &
	\\
	\hline
	PROJECT &
	Produces a new relation with only some of the attributes of $R$, and removes duplicate tuples. &
	\\
	\hline
	THETA JOIN &
	Produces all combinations of tuples from $R_1$ and $R_2$ that satisfy the join condition. &
	\\
	\hline
	EQUIJOIN &
	Produces all combinations of tuples from $R_1$ and $R_2$ that satisfy a join condition with only equality comparisons. &
	\\
	\hline
	NATURAL JOIN &
	Same as EQUIJOIN except that the join attributes of $R_2$ are not included in the resulting relation; if the join attributes have the same names, they do not have to be specified at all. &
	\\
	\hline
	UNION &
	Produces a relation that includes all the tuples in $R_1$ or $R_2$ or both $R_1$ and $R_2$; $R_1$ and $R_2$ must be union compatible. &
	\\
	\hline
	INTERSECTION &
	Produces a relation that includes all the tuples in both $R_1$ and $R_2$; $R_1$ and $R_2$ must be union compatible. &
	\\
	\hline
	DIFFERENCE &
	Produces a relation that includes all the tuples in $R_1$ that are not in $R_2$; $R_1$ and $R_2$ must be union compatible. &
	\\
	\hline
	CARTESIAN PRODUCT &
	Produces a relation that has the attributes of $R_1$ and $R_2$ and includes as tuples all possible combinations of tuples from $R_1$ and $R_2$. &
	\\
	\hline
	DIVISION &
	Produces a relation $R(X)$ that includes all tuples $t[X]$ in $R_1(Z)$ that appear in $R_1$ in combination with every tuple from $R_2(Y)$, where $Z = X \cup Y$. &
	\\
	\hline
\end{tabular}
}
\newpage
% TODO: \usepackage{graphicx} required
\includegraphics[width=1\linewidth]{figures/2-15}
% TODO: \usepackage{graphicx} required
\includegraphics[width=1\linewidth]{../figures/2-16}
% TODO: \usepackage{graphicx} required
\includegraphics[width=1\linewidth]{../figures/2-17}
% TODO: \usepackage{graphicx} required
\includegraphics[width=1\linewidth]{../figures/2-18}
% TODO: \usepackage{graphicx} required
\includegraphics[width=1\linewidth]{../figures/2-19}
% TODO: \usepackage{graphicx} required
\includegraphics[width=1\linewidth]{../figures/2-20}
% TODO: \usepackage{graphicx} required
\includegraphics[width=1\linewidth]{../figures/2-21}
% TODO: \usepackage{graphicx} required
\includegraphics[width=1\linewidth]{../figures/2-22}
% TODO: \usepackage{graphicx} required
%\includegraphics[width=1\linewidth]{../figures/2-23}
% TODO: \usepackage{graphicx} required
\includegraphics[width=1\linewidth]{../figures/2-24}
% TODO: \usepackage{graphicx} required
\includegraphics[width=1\linewidth]{../figures/2-25}
% TODO: \usepackage{graphicx} required
\includegraphics[width=1\linewidth]{../figures/2-26}
% TODO: \usepackage{graphicx} required
\includegraphics[width=1\linewidth]{../figures/2-27}
% TODO: \usepackage{graphicx} required
\includegraphics[width=1\linewidth]{../figures/2-28}
% TODO: \usepackage{graphicx} required
\includegraphics[width=1\linewidth]{../figures/2-29}
% TODO: \usepackage{graphicx} required
\includegraphics[width=1\linewidth]{../figures/2-30}
% TODO: \usepackage{graphicx} required
\includegraphics[width=1\linewidth]{../figures/2-31}
% TODO: \usepackage{graphicx} required
\includegraphics[width=1\linewidth]{../figures/2-32}
% TODO: \usepackage{graphicx} required
\includegraphics[width=1\linewidth]{../figures/2-33}
% TODO: \usepackage{graphicx} required
\includegraphics[width=1\linewidth]{../figures/2-34}
% TODO: \usepackage{graphicx} required
\includegraphics[width=1\linewidth]{../figures/2-35}
% TODO: \usepackage{graphicx} required
\includegraphics[width=1\linewidth]{../figures/2-36}
% TODO: \usepackage{graphicx} required
\includegraphics[width=1\linewidth]{../figures/2-37}
% TODO: \usepackage{graphicx} required
\includegraphics[width=1\linewidth]{../figures/2-38}
% TODO: \usepackage{graphicx} required
\includegraphics[width=1\linewidth]{../figures/2-39}
% TODO: \usepackage{graphicx} required
\includegraphics[width=1\linewidth]{../figures/2-40}
\cheatbox{Ch3. Relational Data Model}{\textStrech}{
\textbf{ER-to-Relational Mapping Algorithm}
\begin{enumerate}
    \item \textbf{Mapping of Regular Entity Types:} 
    Create a relation $R$ for each regular entity type. Include all simple attributes. Choose one key attribute as the Primary Key ($PK$).

    \item \textbf{Mapping of Weak Entity Types:} 
    Create a relation $R$ with all simple attributes of the weak entity. Include the $PK$ of the owner entity as a Foreign Key ($FK$). 
    \textit{PK of R} = $FK_{Owner}$ + Partial Key of the weak entity.

    \item \textbf{Mapping of Binary 1:1 Relationship Types:} 
    \begin{itemize}
        \item \textit{Foreign Key Approach:} Add the $PK$ of one entity (preferably the one with total participation) as a $FK$ in the other relation.
        \item \textit{Merged Relation Approach:} Combine both entities into a single relation (possible if both participations are total).
        \item \textit{Cross-Reference Approach:} Create a separate relationship relation with foreign keys to both entities.
    \end{itemize}

    \item \textbf{Mapping of Binary 1:N Relationship Types:} 
    Identify the relation at the \textbf{N-side}. Add the $PK$ of the \textbf{1-side} relation as a $FK$ in the N-side relation.

    \item \textbf{Mapping of Binary M:N Relationship Types:} 
    Create a new relation $S$ to represent the relationship. Include the $PKs$ of both participating entities as $FKs$ in $S$. The combination of these $FKs$ forms the $PK$ of $S$.

    \item \textbf{Mapping of Multivalued Attributes:} 
    Create a new relation $R$. Include the multivalued attribute and the $PK$ of the parent entity as a $FK$. 
    \textit{PK of R} = The combination of the attribute value and the $FK$.

    \item \textbf{Mapping of N-ary Relationship Types ($n > 2$):} 
    Create a new relation $S$. Include the $PKs$ of all participating entity types as $FKs$. The $PK$ of $S$ is usually the combination of these foreign keys.

    \item \textbf{Mapping of Specialization or Generalization:} 
    Convert based on constraints (Disjoint/Overlapping, Total/Partial):
    \begin{itemize}
        \item \textit{Multiple relations:} Superclass table + Subclass tables (with Superclass PK).
        \item \textit{Multiple relations:} Subclass tables only (containing all attributes).
        \item \textit{Single relation:} One table with a type discriminator attribute.
        \item \textit{Single relation:} One table with multiple boolean flags.
    \end{itemize}

    \item \textbf{Mapping of Union Types (Categories):} 
    Create a surrogate key for the category relation. Add this surrogate key as a $FK$ to each superclass relation.
\end{enumerate}
}

