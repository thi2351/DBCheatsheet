\cheatbox{Ch5. Functional Dependencies \& Normalization}{\textStrech}{
	
\subsubsection*{Functional Dependencies}


\textbf{1. Direct dependency (Fully functional dependency):}  
All non-key attributes are fully functionally dependent on the primary key, i.e.,
the primary key determines all other attributes.
\[
\text{SSN} \rightarrow \{\text{Name, BDate, Address, DNO}\}
\]

\textbf{Indirect dependency (Transitive dependency):}  
A non-key attribute depends on the primary key through another non-key attribute.
\[
\text{SSN} \rightarrow \text{DNO}, \quad \text{DNO} \rightarrow \text{DName}
\]

\textbf{Partial dependency:}  
A non-key attribute depends on only part of a composite primary key, not the entire key.
\[
\{\text{SSN, PNumber}\} \text{ is a composite key}
\]
\[
\text{SSN} \rightarrow \text{EName}, \quad
\text{PNumber} \rightarrow \{\text{PName, PLocation}\}
\]

\subsubsection*{2. Inference Rules}
\textbf{IR1. Reflexivity}:  
If $Y \subseteq X$, then $X \rightarrow Y$.

\textbf{IR2. Augmentation}:  
If $X \rightarrow Y$, then $XZ \rightarrow YZ$  
(where $XZ$ denotes $X \cup Z$).

\textbf{IR3. Transitivity}:  
If $X \rightarrow Y$ and $Y \rightarrow Z$, then $X \rightarrow Z$.

\textbf{IR4. Decomposition}:  
If $X \rightarrow YZ$, then $X \rightarrow Y$ and $X \rightarrow Z$.

\textbf{IR5. Union}:  
If $X \rightarrow Y$ and $X \rightarrow Z$, then $X \rightarrow YZ$.

\textbf{IR6. Pseudotransitivity}:  
If $X \rightarrow Y$ and $WY \rightarrow Z$, then $WX \rightarrow Z$.
}

\cheatbox{Ch5. Functional Dependencies \& Normalization}{\textStrech}{
\subsubsection*{1NF}
A relation is in \textbf{1NF} if each attribute contains only \textbf{atomic (indivisible) values}, i.e., each row--column intersection
has exactly one value.
Composite attributes, multivalued attributes, and nested relations are not allowed.

\includegraphics[width=1\linewidth]{figures/2-43}
\subsubsection*{2NF}
A relation is in \textbf{2NF} if it is in 1NF and
\textbf{all non-prime attributes are fully functionally
	dependent on the entire primary key}.
2NF eliminates \textbf{partial dependencies} by decomposing
the relation based on partial keys.

\includegraphics[width=1\linewidth]{figures/2-44}

}

\cheatbox{Ch5. Functional Dependencies \& Normalization}{\textStrech}{
	\subsubsection*{3NF}
A relation schema $R$ is in \textbf{3NF} if it is in 2NF and \textbf{no non-prime attribute is transitively dependent on the primary key}.
	\includegraphics[width=1\linewidth]{figures/2-45}
	\includegraphics[width=1\linewidth]{figures/2-46}
	\subsubsection*{BCNF}
A relation schema $R$ is in \textbf{BCNF} if for every non-trivial functional
dependency $X \rightarrow Y$ in $R$, \textbf{$X$ is a superkey}.

BCNF is a stricter form of 3NF and eliminates all anomalies caused by
functional dependencies, even those involving candidate keys.

}
\cheatbox{Ch5. Functional Dependencies \& Normalization}{\textStrech}{
	\subsubsection*{1. Algorithm to find minimal cover F for a set of FDs E}
	\includegraphics[width=1\linewidth]{figures/2-49}
	
	\subsubsection*{2. Algorithm to decompose to 3NF}
	\includegraphics[width=1\linewidth]{figures/2-48}
	
	\subsubsection*{3. Algorithm to decompose to BCNF}
	\includegraphics[width=1\linewidth]{figures/2-47}
}

