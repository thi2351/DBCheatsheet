\cheatbox{Ch1. An Overview of Database Systems}{\textStrech}{
\subsubsection*{1. Databases and File Processing Systems}
\begin{tabular}{|p{3.5cm}|p{3.5cm}|}
\hline
\textbf{File Processing Systems} & \textbf{Database Systems} \\
\hline
Data are stored separately for each application & Data are organized centrally in a database \\
\hline
Uncontrolled data redundancy & Controlled data redundancy \\
\hline
Programs are tightly coupled with data structures & Program--data independence is ensured \\
\hline
Difficult to maintain and extend when requirements change & Easy to maintain and extend \\
\hline
Data sharing among applications is difficult & Data sharing is supported for multiple users and applications \\
\hline
\end{tabular}

\subsubsection*{2. Three-Schema Architecture and Data Independence}
A database system adopts the three-schema architecture:
\begin{itemize}[leftmargin=0.5cm, itemsep=0pt, topsep=0pt]
    \item \textbf{External level}: user views tailored to different user groups.
    \item \textbf{Conceptual level}: a complete logical description of the entire database.
    \item \textbf{Internal level}: the physical storage structure of the database.
\end{itemize}

\includegraphics[width=1\linewidth]{figures/1-1.png}
}