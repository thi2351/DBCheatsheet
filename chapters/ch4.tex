\cheatbox{Ch4. SQL Language}{\textStrech}{
	\subsubsection*{1. SQL Language}
\begin{itemize}
	\item \textbf{Data Definition Language (DDL)}: used to define and modify database schema  
	(\texttt{CREATE}, \texttt{ALTER}, \texttt{DROP})
	
	\item \textbf{Data Manipulation Language (DML)}: used to retrieve and manipulate data  
	(\texttt{SELECT}, \texttt{INSERT}, \texttt{DELETE}, \texttt{UPDATE})
	
	\item \textbf{Data Control Language (DCL)}:
	\begin{itemize}
		\item \textit{Access control}: \texttt{GRANT}, \texttt{REVOKE}
		\item \textit{Transaction control}: \texttt{COMMIT}, \texttt{ROLLBACK}, \texttt{SET AUTOCOMMIT OFF}
	\end{itemize}
\end{itemize}

\textbf{Note:} SQL is \textbf{case-insensitive} and is often \textbf{extended by DBMSs} with additional proprietary features.
\subsubsection*{2. DDL Language}
\textbf{CREATE SCHEMA} is used to define a logical container (namespace) that groups
database objects such as tables, views, domains, and constraints.
It helps organize the database, avoid name conflicts, and support authorization. 

Syntax: \texttt{CREATE SCHEMA schema\_name} \texttt{[AUTHORIZATION user\_name];}


\textbf{CREATE TABLE} defines the structure of a relation, including attributes,
data types, and integrity constraints.

Syntax:

\texttt{CREATE TABLE table\_name (} \\
\texttt{\quad attribute\_name data\_type [attribute\_constraint],} \\
\texttt{\quad ...} \\
\texttt{\quad [table\_constraint]} \\
\texttt{);}


\textbf{CREATE DOMAIN} is used to define a user-defined data type with optional
constraints that can be reused across multiple tables. It improves consistency
and reduces redundancy in schema definitions.

\textit{Syntax:}

\texttt{CREATE DOMAIN domain\_name AS data\_type} \\
\texttt{\quad [DEFAULT default\_value]} \\
\texttt{\quad [CHECK (VALUE condition)];}

\begin{itemize}[leftmargin=*, itemsep=0pt, topsep=0pt]
	\item Domain constraints are enforced wherever the domain is used.
	\item \texttt{VALUE} refers to the attribute value being checked.
	\item Changing a domain affects all attributes defined using that domain.
\end{itemize}
}

\cheatbox{Ch4. SQL Language}{\textStrech}{
\textbf{Attribute-level constraints}:
\begin{itemize}[leftmargin=*, itemsep=0pt, topsep=0pt]
	\item \texttt{NOT NULL}: disallows NULL values for the attribute.
	\item \texttt{DEFAULT value}: assigns a default value when none is provided.
	\item \texttt{UNIQUE}: ensures all values in the attribute are distinct.
	\item \texttt{CHECK(condition)}: restricts attribute values to satisfy a condition.
	\item \texttt{PRIMARY KEY}: uniquely identifies each tuple and disallows NULLs.
	\item \texttt{REFERENCES table(attribute)}: enforces referential integrity with a referenced table.
\end{itemize}

\textbf{Table-level constraints}:
\begin{itemize}[leftmargin=*, itemsep=0pt, topsep=0pt]
	\item \texttt{PRIMARY KEY (A, B)}: defines a composite primary key over multiple attributes.
	\item \texttt{UNIQUE (A, B)}: enforces uniqueness over a combination of attributes.
	\item \texttt{FOREIGN KEY (A) REFERENCES R(B)}: specifies a foreign key relationship between tables.
	\item \texttt{CHECK(attr\_name condition)}: restricts tuples based on a condition involving multiple attributes.
\end{itemize}

\textbf{Referential actions}:
\begin{itemize}[leftmargin=*, itemsep=0pt, topsep=0pt]
	\item \texttt{ON DELETE CASCADE}: deletes referencing tuples when the referenced tuple is deleted.
	\item \texttt{ON DELETE SET NULL}: sets foreign key values to NULL when the referenced tuple is deleted.
	\item \texttt{ON UPDATE CASCADE}: updates foreign key values when the referenced key is updated.
	\item \texttt{RESTRICT}: prevents deletion or update of a referenced tuple (default behavior).
\end{itemize}


\textbf{ALTER TABLE} is used to modify the structure of an existing table,
including attributes, constraints, and referential actions, without recreating
the table.

\textit{Common operations:}

\begin{tabular}{|p{2.2cm}|p{4.8cm}|}
	\hline
	\textbf{Operation} & \textbf{Syntax} \\
	\hline
	
	Add attribute &
	\texttt{ALTER TABLE table\_name ADD attribute data\_type [constraint];}
	\\
	\hline
	
	Drop attribute &
	\texttt{ALTER TABLE table\_name DROP attribute RESTRICT (default)/CASCADE;}
	\\
	\hline
	
	Modify attribute &
	\texttt{ALTER TABLE table\_name ALTER attribute SET data\_type;}
	\\
	\hline
	
	Add constraint &
	\texttt{ALTER TABLE table\_name ADD CONSTRAINT cname constraint\_definition;}
	\\
	\hline
	
	Drop constraint &
	\texttt{ALTER TABLE table\_name DROP CONSTRAINT cname;}
	\\
	\hline
	
	Modify referential action &
	\texttt{ALTER TABLE table\_name ADD FOREIGN KEY (A)} \\
	& \texttt{REFERENCES R(B) ON DELETE CASCADE;}
	\\
	\hline
	
\end{tabular}

\begin{itemize}[leftmargin=*, itemsep=0pt, topsep=0pt]
	\item ALTER TABLE affects only the table structure, not existing data values.
	\item Constraint names are required when dropping constraints.
	\item Some DBMSs restrict dropping or modifying attributes referenced by foreign keys.
\end{itemize}

}

\includegraphics[width=1\linewidth]{figures/2-41}
Note: Có thể tạm hoãn việc kiểm tra ràng buộc cho đến cuối 1 transaction.


\includegraphics[width=1\linewidth]{figures/2-42}
Note: Alter Table lúc drop attribute/constraint cũng có quy tắc tương tự như vậy.

\cheatbox{Ch4. SQL Language}{\textStrech}{
	\subsubsection*{3. DML Language}
	\textbf{SELECT STATEMENT}
	
	
	\texttt{SELECT [DISTINCT | ALL]} 
	\texttt{\quad \{ * | columnExpression [AS newName] | built\_inFunction }
	\texttt{\quad \quad [ , ... ] \}}
	
	\texttt{[FROM TableName | ViewName [alias] [ , ... ]]}
	
	\texttt{[WHERE rowCondition]}
	
	\texttt{[GROUP BY columnList]} 
	
	\texttt{[HAVING groupCondition]}
	
	\texttt{[ORDER BY columnList | columnPositionList [ASC | DESC]]}
	
	\begin{tabular}{|p{1.2cm}|c|p{4.5cm}|}
		\hline
		\textbf{Execution Order} & \textbf{Clause} & \textbf{Meaning} \\
		\hline
		1 & FROM &
		Specifies and joins table(s) or view(s) to be used \\
		\hline
		2 & WHERE &
		Filters rows based on row-level conditions \\
		\hline
		3 & GROUP BY &
		Forms groups of rows having the same grouping column values \\
		\hline
		4 & HAVING &
		Filters groups based on group-level conditions \\
		\hline
		5 & SELECT &
		Specifies the output attributes to be returned \\
		\hline
		6 & ORDER BY &
		Specifies the order of the output \\
		\hline
	\end{tabular}
}